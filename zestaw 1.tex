\documentclass{mwbk}

\usepackage{polski}
\usepackage[utf8]{inputenc}

\usepackage{mdframed}
\usepackage{amsthm}
\usepackage{amsmath}
\usepackage{amssymb}
\usepackage{url}



\newtheorem{zad}{Zadanie}[chapter]


\begin{document}
\title{Nowa Dyskretna I}
\date{\today}
\maketitle

Hejka!

W związku z reworkiem Dyskretnej I jakiego doświadczyliśmy w zeszłym
roku -- a przez to unieważnienieniem zbieranej przez lata bazy (i
fenomenalnych notatek Jakuba Żuchowskiego) postanowiłem chociaż
spróbować coś zmienić w tej materii. Tak więc oto tutaj zaczyna się
świadectwo tej próby. \newline


Na start przyznam szczerze, że jest to projekt baaaaardzo w przebudowie,
dopiero zaczynam zabawę z \LaTeX --em, mam mocno średni wpm, a o skillsach
z dyskretnej nie wspomnę... -- więc tu się zwracam z prośbą o pomoc do
Ciebie, drogi czytelniku!\newline

Liczę, że wspólnymi siłami uda nam się stworzyć coś, co przyda się
nie tylko przyszłym rocznikom -- ale i nam podczas nauki do kolokwiów
i ćwiczeń. Tak więc jeżeli masz jakieś uwagi i poprawki możesz
się z nimi ze mną podzielić na Discordzie pod nickiem \url{ssk12.}
lub na kanale dyskretnej. Tak samo własne rozwiązania bardzo ułatwią mi pracę.
Jeżeli będziesz chciał, możesz zostać wspomniany jako autor rozwiązania.\newline

Link do repo na githubie, gdzie znajduje się aktualna wersja tego pliku:
\url{https://github.com/ssk12o/Nowa-Dyskretna-I} \newline

Zadania są kodwane w następujący sposób:
\begin{itemize}
    \item Ć -- chaotyczne notatki z ćwiczeń, esencja \textit{work in progress},
    \item W - wskazówka, szkic rozwiązania, niepełne rozwiązanie (nie odpowidam za jakość),
    \item A -- rozwiązanie autorskie,
    \item AR `Jan Kowalski' -- rozwiązanie dzi,
    \item BRAK -- oczywiste. Zachęcam jednak do podesłania własnego rozwiązania.
\end{itemize}

\tableofcontents

\chapter{Zestaw 1}


\begin{zad}[W]
    Na płaszczyźnie poprowadzono $n$ prostych, z których żadne dwie nie
    są równoległe i żadne trzy nie przechodzą przez ten sam punkt.
    Wyznacz liczbę:
    \begin{enumerate}
        \item obszarów, na które te proste dzielą płaszczyznę;
        \item obszarów ograniczonych, na które te proste dzielą płaszczyznę.
    \end{enumerate}
\end{zad}
\begin{proof}
    Podpunkty:
    \begin{enumerate}
        \item Warto zauważyć, że ze względu na warunki z treści
              wiemy, że każda $n$-ta nowa prosta którą dokładamy przecina każdą z $n-1$
              prostych które już istnieją -- i tworzy $n$ nowych obszarów. Tak więc
              $R(n)= R(n-1) + n$. Starczy rozwalić rekurencję.
        \item Podobnie.
    \end{enumerate}
\end{proof}

\begin{zad}[W]
    Ciąg Fibonacciego $\{F_n\}_{n \in \mathbb{N}}$ zadany jest przez
    $F_0=0$, $F_1=1$ i $F_{n+2}=F_{n+1}+F_n$. Udowodnij, że: \\
    \begin{enumerate}
        \item $F_0 + ... + F_n = F_{n+2} - 1$;
        \item $5|F_{5n}$,
        \item $F_n < 2_n$.
    \end{enumerate}
\end{zad}
\begin{proof}
    Wszystkie podpunkty można łatwo dowieść indukcyjnie.
\end{proof}

\begin{zad}[BRAK]
    Turniej $n$-wierzchołkowy to dowolny graf skierowany $G = (V, E)$, gdzie $|V| = n$
    i w którym $(u, v) \in E$ lub $(v, u) \in E$ dla dowolnych $u, v \in V$.
    Pokaż, że w dowolnym niepustym turnieju istnieje wierzchołek z którego można “przejść”
    po krawędziach zgodnie z ich skierowaniem do dowolnego innego wierzchołka w co
    najwyżej dwóch krokach.
\end{zad}
\begin{proof}
\end{proof}

\begin{zad}[Ć]
    %Ni chuja nie mam pojęcia o co chodzi, nie miałem teorii grafów i nie czytałem o nich. \\
    Udowodnij, że każdy turniej ma ścieżkę Hamiltona.
\end{zad}
\begin{proof}
    Dowód indukcyjny.
    \begin{enumerate}
        \item przypadek trywialny
        \item z założenia indukcyjnego wiemy że ten graf $G' = (V', E')$ ma
              on ścieżkę Hamiltona. Nazwijmy ją $P = v_1 ... v_n$. Rozważmy:
              \begin{enumerate}
                  \item $(v, v_1) \in E$. Wtedy $P = v v_1 v_2 ... v_n$
                  \item $(v_n, v)$ rys $P = v_1 v_2 ... v_n v$
                  \item trzeci przypadek rys. Niech $i \in \{1, 2, 3, ..., n\}$
                        będzie najmniejszym indeksem $(v_i, v) \in E$ oraz $(v, v_{i+1}) \in E$,
                        wtedy $P = v_1 v_2 ... v_i v v_{i+1} v_{i+2} ... v_n$
              \end{enumerate}
    \end{enumerate}
\end{proof}





\begin{zad}[W]
    W każdym polu szachownicy rozmiaru $n x n $ znajduje się jedna osoba.
    Część osób zarażona jest wirusem grypy. Wirus grypy rozprzestrzenia się w dyskretnych
    odstępach czasowych w sposób następujący:
    \begin{itemize}
        \item osoby zarażone pozostają zarażone,
        \item osoba ulega zarażeniu jeżeli co najmniej dwie sąsiadujące z nią osoby są już zarażone
              (przez osobę sąsiednią rozumiemy osobę siedzącą z przodu, z tyłu, z lewej lub prawej
              strony).
              Wykaż, że jeżeli na początku zarażonych jest istotnie mniej niż n osób, to w każdej chwili
              przynajmniej jedna osoba pozostaje niezarażona.
    \end{itemize}
\end{zad}

\begin{proof}
    Jakaś podpowiedź. Przekątne (małe) dzielą plansze po zarażeniu
    na kwadraty zarażonych. Przerwy w ciągach diagonalnych robią przerwy na końcu.

    Ewentualnie:
    Jeżeli rozrzuick komórki w pierwszej iteracji, jesti ich n-1. liczymi obwod
    tej figory. obwod ten jest rowny maksymalnie 4n - 4

    jesli tylko uzasadnic formalnie ze przy kazdej iteracji obwod
    jest conajwyzej taki ajk poprzednio, co tez znaczy ze nie
    wszystkie sa zarażone, bo obwod byłby równy 4n.

    Jak udowodnić tą zależność?

    Od bartka:
    Pusta kratka, $n$ zarażonych obok i $n \in \{1, 2, 3, 4\}$.
    Rozważamy przypadki ile krawędzi jest przy każdym $n$.
    Patrzymy ile znika i ile powstaje. Dochodzimy do tego że jest
    nie więcej.
\end{proof}






\begin{zad}[BRAK]
    Wykaż, że w grupie $n$ osób istnieją dwie, które mają taka samą liczbę znajomych.
\end{zad}
\begin{proof}
\end{proof}

\begin{zad}[Ć]
    Przy okrągłym stole jest $n$ miejsc oznaczonych proporczykami różnych
    państw. Ambasadorowie tych państw usiedli przy tym stole tak, że żaden z nich nie siadł
    przy właściwym proporczyku. Wykaż, że można tak obrócić stołem, że co najmniej 2
    ambasadorów znajdzie się przed proporczykiem swojego państwa.
\end{zad}
\begin{proof}
    Niech $f:[n] \to [n]$ tak że $\forall_{k\in[n]} f(x) \neq k$.
    $f$ - permutacja
    Niech $g$ - permutacja g = (1 2 3 4 5 ... n)
    $O_n = f \circ g \circ g \circ g$ (i g złożone n razy)

    Dirichlet. $\forall f:[n] \to [n-1]$ istnieje $x_1, x_2$
    że $x_1\neq x_2$ tże $f(x_1)=f(x_2)$

    $\forall_{k \in [n]} \exists_{i \in [n-1]} O_i(k) = k$
    $\leftarrow$ funkcja by wrócił do właściwego proporczykami.

    $F(k)$ = ilość obrotów stołem aby ambasador $k$ miał
    przed sobą swój proporczyk.
    Widzimy że nowa funkcja $F(k): [n] \to [n-1]$
    Z zasady Dirichleta istnienieje dwóch ambasadorów $x_1, x_2$ tże $x_1 \neq x_2$
    tże $F(x_1) = F(x_2)$.
\end{proof}


\begin{zad}[Ć]
    Pokaż, że w dowolnym ciągu n liczb całkowitych istnieje (niepusty)
    podciąg kolejnych elementów taki, że suma wyrazów podciągu jest wielokrotnością n.
\end{zad}
\begin{proof}
    Weźmy ciąg: $a_1, a_2, a_3 ..., a_n$

    Wprowadzmy ciag sum częściowych:
    $S_k = a_1 + a_2 + ... + a_k$

    Reszty z dzielenia przez $n$ są następującej postaci
    $\{0, 1, 2, ... n_1\}$.
    jesli istnije $S_k$ tże skmodn=o to koniec zadania

    zalozmy ze tak nie jest, czyli ze dla kazdego ka w [n]
    $S_k$ $mod n  \neq 0$
    Czyli ile możliwości mamy na resztę od 1 do n-1.
    z zasady diricchleta widzimy ze istnieje k'<k tże
    Sk i Sk' maja tkaa sama reszte z dzielenia.
    ale wtedy Sk-Sk' jest podzilene przez n i
    reprezentuje sume ak'+1 + ak'+2 +...+ak

    Widzimy zatem ze ciag ak'+1 do ak ma żadaną własność

    jeżeli weżmiemy jakiś pierwszy podciąg
    $a_1 modn \neq 0$
    $a_1 + a_2 mod n \neq 0$ i $a_1 + a_2 mod n$

\end{proof}





\begin{zad}[BRAK]
    Rozważ dowolną rodzinę podzbiorów zbioru $n$--elemetowego zawierającą
    więcej niż połowę wszystkich podzbiorów. Wykaż, że w tej rodzinie muszą być dwa zbiory
    takie, że jeden zawiera się w drugim.
\end{zad}
\begin{proof}
\end{proof}



\begin{zad}[BRAK]
    Dla $n$--elementowego zbioru $X$ rozważ pewną rodzinę jego podzbiorów
    $\mathcal{F}$, gdzie $|F| > n/2$ dla każdego $F \in \mathcal{F}$. Wykaż, że istnieje
    $x \in X$ należący do co najmniej połowy zbiorów z $\mathcal{F}$.
\end{zad}
\begin{proof}

\end{proof}


\begin{zad}[Ć]
    Dana jest kwadratowa szachownica $2n \times 2n$ z wyciętym jednym polem.
    Wykaż, że dla wszystkich wartości $n \geq 1$ możemy pokryć tę szachownicę kostkami w
    kształcie litery L (czyli kwadrat $2 \geq 2$ bez jednego pola).
\end{zad}
\begin{proof}
    Dowód indukcyjny.
    \begin{enumerate}
        \item $n = 1$; rysunek 1
        \item zał ind spełnione dla $n$. teza ind dla $n+1$. rysunek2 ogólnie dzielimy na
              cztery, zał ind, od pozostałych trzech odejmujemy po jednym kawałeczku, załatwiamy
              je z ind a kawałeczekki załatwiamy jednym l-em
    \end{enumerate}
\end{proof}

\begin{zad}[Ć]
    Dana jest kwadratowa szachownica $n \times n$. Dla jakich wartosci $n\geq 1$
    możemy pokryć tę szachownicę kostkami wielkości $2 \times 2$ oraz $3 \times 3$.
\end{zad}
\begin{proof}
    Dla kostek 2x2 i 3x3 oczywiste. Dla ich wielokrotności oczywsite.

    Teraz łączenie. I go nie zrozumiałem. Well, shit happens.
    Ale do przemyślenia.
\end{proof}


\begin{zad}[Zadanie 10]

\end{zad}
\begin{proof}
    zrobione, ale średnio z zapisem.
    Nie zrozumiałem, ale za to macie zdjątko tablicy.
    rysunek1.
    praca domowa
\end{proof}



















\chapter{Zestaw 2}
\begin{zad}[W]
    Na ile sposobów można ustawić $n$ wież na szachownicy
    $n \times n$ tak, by żadne dwie nie znajdowały się w
    polu wzajemnego rażenia.
\end{zad}
\begin{proof}
    Starczy zauważyć, że dla każdej wieży wybieramy rząd i kolumnę
    w której się znajduje -- i tym samym zmniejsza liczbę dostępnych
    o jeden. Tak więc odpowiedź wynosi: \[n \cdot n \cdot (n-1) \cdot
        (n-1) \cdot ... \cdot 2 \cdot 2 \cdot 1 \cdot 1 =  n! \cdot n!\]
\end{proof}

\begin{zad}[W]
    Na ile sposobów można ustawić $k$ wież na szachownicy $n \times m$
    tak, by żadne dwie nie znajdowały się w polu wzajemnego rażenia.
\end{zad}
\begin{proof}
    Zadanie analogiczne od poprzedniego - z tym, że zmienił nam się
    rozmiar planszy, a ponadto nie wypełniamy jej całej. Zasada
    pozostaje jednak ta sama. Na start jednak warto założyć, że
    $k \leq max\{n, m\}$ (choć w sumie jeżeli tak nie jest, to
    odpowiedź to 0). Mając to już za sobą.
    \[n \cdot m \cdot (n-1) \cdot (m-1) \cdot ... \cdot (n - k +1) \cdot (m -k +1)\]
    (wykonujemy mnożenie $k + k$ elementów -- stąd to $-k + 1$).
\end{proof}

\begin{zad}
    Znalźć definicje rekurencyjne następujących ciągów:
    \begin{enumerate}
        \item $a(n)$ -- liczba słów długości $n$ nad alfabetem
              $\{0, 1\}$, które nie zawierają dwóch jedynek koło siebie.
        \item  $b(n)$ -- liczba różnych pokryć prostokąta o wymiarze
              $2 \times n$ dominami wymiaru $2 \times 1$.
    \end{enumerate}
\end{zad}
\begin{proof}
    \begin{enumerate}
        \item Oczywiście $a(1)=2$, $a(2)=3$.  Rozważmy słowo n elementowe.
              Zauważamy, że jeżeli ono kończy sie ono zerem to poprzedzające słowo $n-1$
              elementowe jest dowolne. Jeżeli nastomiast kończy się jedynką,
              to poprzedzające słowo $n-2$ elementowe jest dowolne
              (tak jakby sie cofamy krok dalej by mieć dowolność).
              Stąd: $a(n)=a(n-1)+a(n-2)$.
        \item Analogicznie do poprzedniego. Jak wiemy $a(1) = 1$, $a(2) = 2$. Zastanówmy się nad $a(n)$:
              Rozważamy ciąg o dłuugości $n$. Jeżeli na końcu jest blok poziomy,
              to wiemy że powstał on z ciągu długości $a(n-2)$.
              Jeżeli jest pionowy, to widzmy że musiał on powstać z ciągu długości $n-1$.
              A stąd $a(n) = a(n-1) + a(n-2)$.
    \end{enumerate}
\end{proof}

\begin{zad}[W]
    Ile rozwiązań ma równanie $x_1 + x_2+x_3+x_4 = 7$:
    \begin{enumerate}
        \item gdzie $x_i$ są liczbami naturalnymi?
        \item gdzie $x_i$ są dodatnimi liczbami naturalnymi?
    \end{enumerate}
\end{zad}
\begin{proof}
    Starczy wykonać siatkę która na dole ma $x_1 ... x_4$, a w rzędach
    wartości od 0 (1) do 7. Wysokość nad $x_i$ oznacza wartość sumy do
    tego elementu włącznie, a przeskok względem poprzedniej wysokości
    oznacza wartość danego $x_i$. Więc można to potraktować jako ścieżki,
    a na mocy własności z wykładu wiemy ile jest ścieżek po kracie. Tak więc:
    \begin{enumerate}
        \item \[\binom{7+4-1}{7}=\binom{10}{7}\]
        \item \[\binom{3+4-1}{3}=\binom{6}{3}\]
    \end{enumerate}
\end{proof}



\begin{zad}
    Rozważmy czekoladę złożoną z $m\times n$ kostek.
    Na ile sposobów można wykroić prostokąt złożony z $k \times k$
    sąsiadujących ze sobą kostek ze sobą kostek czekolady?
\end{zad}
\begin{proof}
    Na początek warto wykonać założenia $k \leq n$, $k \leq m$
    (albo stwierdzić że wtedy odpowiedź to zero). Potem
\end{proof}


\begin{zad}
    (Reguła sumowania po górnym indeksie). Udowodnij, że dla
    $n, k \in \mathbb{N}$ zachodzi
    \[\sum_{j=0}^k\binom{j}{k} = \binom{n+1}{k+1}\]
\end{zad}
\begin{proof}
    Starczy zrobić indukcje. Wychodzi za darmo.
    Kombinarotrycznie też się da.
\end{proof}

\begin{zad}
    (Reguła sumowania równoległego). Udowodnij, że dla $n, k \in \mathbb{N}$
    zachodzi \[     \]
\end{zad}\
\begin{proof}
    Starczy zrobić indukcje. Wychodzi za darmo.
\end{proof}

\begin{zad}
    Ile jest funkcji $f:\{1, ..., n\} \to \{1, ..., n\}$ monotonicznych takich,
    że $f(i) \leq f(j) $ dla $i < j$?
\end{zad}
\begin{proof}
    Rozwiązanie kożystające z tej samej koncepcji co 4 z tego zestawu.
    Tworzymy kratkę, gdzie kolumny to argumenty, a rzędy to wartości.
    Interesują nas ścieżki z początku do końca - które generują nam wszystkie funkcje
    spełniające warunki zadania. Tak więc:
    \[ \binom{n + n -1 }{n} = \binom{2n-1}{n} \].
\end{proof}


\begin{zad}
    Ile jest $k$--elementowych podzbiorów zbioru $n$--elementowego, które nie
    zawierają dwóch sąsiednich liczb?
\end{zad}
\begin{proof}
    Jest nie jest to funkcja 1 - 0 (funkcaj charakterystyczna). Generujemy
    zbiór typu 10101 aż do uzyskania k jedynek (tu k=3), ustalamy miejsca
    w które możemy dopełniać zerami (by ciągi były długości n - funkcja
    charakterystyczna). Czyli mamy k-1 zer na start, n-k-k+1 zer jeszcze do zrobienia


    Czyli finalnir
    \[\binom{n-k+1}{k}  \]
\end{proof}


























\end{document}