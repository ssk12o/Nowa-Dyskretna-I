\documentclass{mwbk}


\usepackage{polski}
\usepackage[utf8]{inputenc}

\usepackage{amsthm}
\usepackage{amsmath}
\usepackage{amssymb}
%\usepackage{url}
\usepackage[nobreak=true]{mdframed}
\usepackage{hyperref}
\usepackage{graphics}


%           środowiska numerowane
\newtheorem{zad}{Zadanie}[chapter]


%           nagłówki i stopki
%\usepackage{fancyhdr}
%\pagestyle{fancy}
%\fancyhf{}
%
%\fancyhead[L]{\includegraphics[height=0.666.com]{logo_uczelni.png}}
%\fancyhead[R]{Nowa Dyskretna 1}
%\fancyfoot[R]{\thepage}






%           strona tytułowa



\begin{document}
\title{Nowa Dyskretna I}
\author{}
\date{\today}
\maketitle

\begin{center}
    \textbf{\href{https://drive.google.com/drive/folders/1ma90kiyWDu3Lr4yZNFwNr48FS1fOthvE}{Aktualna wersja pliku znajduje sie tutaj! }}
\end{center}



Hejka!

W związku z reworkiem Dyskretnej I jakiego doświadczyliśmy w zeszłym
roku -- a przez to unieważnienieniem zbieranej przez lata bazy (i
fenomenalnych notatek Jakuba Żuchowskiego) postanowiłem chociaż
spróbować coś zmienić w tej materii. Tak więc oto tutaj zaczyna się
świadectwo tej próby. \newline


Na start przyznam szczerze, że jest to projekt baaaaardzo w przebudowie,
dopiero zaczynam zabawę z \LaTeX --em, mam mocno średni wpm, a o skillsach
z dyskretnej nie wspomnę... -- więc tu się zwracam z prośbą o pomoc do
Ciebie, drogi czytelniku!\newline

Liczę, że wspólnymi siłami uda nam się stworzyć coś, co przyda się
nie tylko przyszłym rocznikom -- ale i nam podczas nauki do kolokwiów
i ćwiczeń. Tak więc jeżeli masz jakieś uwagi i poprawki możesz
się z nimi ze mną podzielić na Discordzie, gdzie znajdziesz mnie pod nickiem \url{ssk12o}
lub na kanale dyskretnej. Tak samo Twoje rozwiązania bardzo ułatwią mi pracę.
Możesz je wysłać mi na priv, albo -- co preferowane, wrzucić do odpowiedniego folderu
na moim dysku google (\href{https://drive.google.com/drive/folders/1ma90kiyWDu3Lr4yZNFwNr48FS1fOthvE?usp=sharing}{tutaj!}).
Plik nazwij numerem zadania, a także, jeżeli chcesz być wspomniany jako autor rozwiązania -- w cudzysłowach wpisz swoje imie/nick. \newline

%Link do repo na githubie, gdzie znajduje się aktualna wersja tego pliku:
%\url{https://github.com/ssk12o/Nowa-Dyskretna-I} \newline


Zadania są kodowane w następujący sposób:
\begin{itemize}
    \item W     - chaotyczne notatki z ćwiczeń, wskazówka, szkic rozwiązania, niepełne rozwiązanie (nie odpowiadam za jakość),
    \item R    - rozwiązanie raczej dobre,
    \item BRAK  - oczywiste. Zachęcam jednak do podesłania własnego rozwiązania,
    \item `Jan Kowalski' - szczególne podziękowanie dla wymienionego.
\end{itemize}

\tableofcontents

\chapter{Zestaw 1}

\textbf{UWAGA! PIERWSZY ZESTAW SKŁADA SIĘ GŁÓWNIE Z (JESZCZE) NIEOBROBIONYCH NOTATEK Z ĆWICZEŃ}


\begin{zad}[W]
    Na płaszczyźnie poprowadzono $n$ prostych, z których żadne dwie nie
    są równoległe i żadne trzy nie przechodzą przez ten sam punkt.
    Wyznacz liczbę:
    \begin{enumerate}
        \item obszarów, na które te proste dzielą płaszczyznę;
        \item obszarów ograniczonych, na które te proste dzielą płaszczyznę.
    \end{enumerate}
\end{zad}
\begin{mdframed}
    Podpunkty:
    \begin{enumerate}
        \item Warto zauważyć, że ze względu na warunki z treści
              wiemy, że każda $n$-ta nowa prosta którą dokładamy przecina każdą z $n-1$
              prostych które już istnieją -- i tworzy $n$ nowych obszarów. Tak więc
              $R(n)= R(n-1) + n$. Starczy rozwalić rekurencję.
        \item Podobnie.
    \end{enumerate}
\end{mdframed}

\begin{zad}[W]
    Ciąg Fibonacciego $\{F_n\}_{n \in \mathbb{N}}$ zadany jest przez
    $F_0=0$, $F_1=1$ i $F_{n+2}=F_{n+1}+F_n$. Udowodnij, że: \\
    \begin{enumerate}
        \item $F_0 + ... + F_n = F_{n+2} - 1$;
        \item $5|F_{5n}$,
        \item $F_n < 2_n$.
    \end{enumerate}
\end{zad}
\begin{mdframed}
    Wszystkie podpunkty można łatwo dowieść indukcyjnie.
\end{mdframed}

\begin{zad}[BRAK]
    Turniej $n$-wierzchołkowy to dowolny graf skierowany $G = (V, E)$, gdzie $|V| = n$
    i w którym $(u, v) \in E$ lub $(v, u) \in E$ dla dowolnych $u, v \in V$.
    Pokaż, że w dowolnym niepustym turnieju istnieje wierzchołek z którego można “przejść”
    po krawędziach zgodnie z ich skierowaniem do dowolnego innego wierzchołka w co
    najwyżej dwóch krokach.
\end{zad}
\begin{mdframed}
\end{mdframed}

\begin{zad}[W]
    %Ni chuja nie mam pojęcia o co chodzi, nie miałem teorii grafów i nie czytałem o nich. \\
    Udowodnij, że każdy turniej ma ścieżkę Hamiltona.
\end{zad}
\begin{mdframed}
    Dowód indukcyjny.
    \begin{enumerate}
        \item przypadek trywialny
        \item z założenia indukcyjnego wiemy że ten graf $G' = (V', E')$ ma
              on ścieżkę Hamiltona. Nazwijmy ją $P = v_1 ... v_n$. Rozważmy:
              \begin{enumerate}
                  \item $(v, v_1) \in E$. Wtedy $P = v v_1 v_2 ... v_n$
                  \item $(v_n, v)$ rys $P = v_1 v_2 ... v_n v$
                  \item trzeci przypadek rys. Niech $i \in \{1, 2, 3, ..., n\}$
                        będzie najmniejszym indeksem $(v_i, v) \in E$ oraz $(v, v_{i+1}) \in E$,\newline
                        wtedy $P = v_1 v_2 ... v_i v v_{i+1} v_{i+2} ... v_n$
              \end{enumerate}
    \end{enumerate}
\end{mdframed}





\begin{zad}[W]
    W każdym polu szachownicy rozmiaru $n x n $ znajduje się jedna osoba.
    Część osób zarażona jest wirusem grypy. Wirus grypy rozprzestrzenia się w dyskretnych
    odstępach czasowych w sposób następujący:
    \begin{itemize}
        \item osoby zarażone pozostają zarażone,
        \item osoba ulega zarażeniu jeżeli co najmniej dwie sąsiadujące z nią osoby są już zarażone
              (przez osobę sąsiednią rozumiemy osobę siedzącą z przodu, z tyłu, z lewej lub prawej
              strony).
              Wykaż, że jeżeli na początku zarażonych jest istotnie mniej niż n osób, to w każdej chwili
              przynajmniej jedna osoba pozostaje niezarażona.
    \end{itemize}
\end{zad}
\begin{mdframed}
    Jakaś podpowiedź. Przekątne (małe) dzielą plansze po zarażeniu
    na kwadraty zarażonych. Przerwy w ciągach diagonalnych robią przerwy na końcu.

    Ewentualnie:
    Jeżeli rozrzucić komórki w pierwszej iteracji, jest ich n-1. liczymy obwod
    tej figury. obwod ten jest rowny maksymalnie 4n - 4

    jesli tylko uzasadnic formalnie ze przy kazdej iteracji obwod
    jest conajwyżej taki ajk poprzednio, co tez znaczy ze nie
    wszystkie sa zarażone, bo obwod byłby równy 4n.

    Jak udowodnić tą zależność?

    Od bartka:
    Pusta kratka, $n$ zarażonych obok i $n \in \{1, 2, 3, 4\}$.
    Rozważamy przypadki ile krawędzi jest przy każdym $n$.
    Patrzymy ile znika i ile powstaje. Dochodzimy do tego że jest
    nie więcej.
\end{mdframed}






\begin{zad}[`Jedrek']
    Wykaż, że w grupie $n$ osób istnieją dwie, które mają taka samą liczbę znajomych.
\end{zad}
\begin{mdframed}
    \textit{Reductio ad absurdum}

    Zakładamy, że każdy ma inną ilość znajomych, czyli jest $n$ osób
    i $n$ możliwych ilości znajomych (od $0$ do $n-1$) -- czyli ilstnieje osoba
    posiadająca $0$ znajomych oraz soba posiadająca $n-1$ znajomych, czyli każdy
    jest jej znajomym -- również osobą posiadająca $0$ znajomych.

    \textit{Sprzeczność!}
\end{mdframed}

\begin{zad}[W]
    Przy okrągłym stole jest $n$ miejsc oznaczonych proporczykami różnych
    państw. Ambasadorowie tych państw usiedli przy tym stole tak, że żaden z nich nie siadł
    przy właściwym proporczyku. Wykaż, że można tak obrócić stołem, że co najmniej 2
    ambasadorów znajdzie się przed proporczykiem swojego państwa.
\end{zad}
\begin{mdframed}
    Niech $f:[n] \to [n]$ tak że $\forall_{k\in[n]} f(x) \neq k$.
    $f$ - permutacja
    Niech $g$ - permutacja g = (1 2 3 4 5 ... n)
    $O_n = f \circ g \circ g \circ g$ (i g złożone n razy)

    Dirichlet. $\forall f:[n] \to [n-1]$ istnieje $x_1, x_2$
    że $x_1\neq x_2$ tże $f(x_1)=f(x_2)$

    $\forall_{k \in [n]} \exists_{i \in [n-1]} O_i(k) = k$
    $\leftarrow$ funkcja by wrócił do właściwego proporczykami.

    $F(k)$ = ilość obrotów stołem aby ambasador $k$ miał
    przed sobą swój proporczyk.
    Widzimy że nowa funkcja $F(k): [n] \to [n-1]$
    Z zasady Dirichleta istnienieje dwóch ambasadorów $x_1, x_2$ tże $x_1 \neq x_2$
    tże $F(x_1) = F(x_2)$.
\end{mdframed}


\begin{zad}[W]
    Pokaż, że w dowolnym ciągu n liczb całkowitych istnieje (niepusty)
    podciąg kolejnych elementów taki, że suma wyrazów podciągu jest wielokrotnością n.
\end{zad}
\begin{mdframed}
    Weźmy ciąg: $a_1, a_2, a_3 ..., a_n$

    Wprowadzmy ciag sum częściowych:
    $S_k = a_1 + a_2 + ... + a_k$

    Reszty z dzielenia przez $n$ są następującej postaci
    $\{0, 1, 2, ... n_1\}$.
    jesli istniej $S_k$ tże skmodn=o to koniec zadania

    załóżmy ze tak nie jest, czyli ze dla kazdego ka w [n]
    $S_k$ $mod n  \neq 0$
    Czyli ile możliwości mamy na resztę od 1 do n-1.
    z zasady diricchleta widzimy ze istnieje k'<k tże
    Sk i Sk' maja tkaa sama reszte z dzielenia.
    ale wtedy Sk-Sk' jest podzielne przez n i
    reprezentuje sume ak'+1 + ak'+2 +...+ak

    Widzimy zatem ze ciag ak'+1 do ak ma żądaną własność

    jeżeli weźmiemy jakiś pierwszy podciąg
    $a_1 modn \neq 0$
    $a_1 + a_2 mod n \neq 0$ i $a_1 + a_2 mod n$

\end{mdframed}





\begin{zad}[`Jedrek' TO BE DONE]
    Rozważ dowolną rodzinę podzbiorów zbioru $n$--elemetowego zawierającą
    więcej niż połowę wszystkich podzbiorów. Wykaż, że w tej rodzinie muszą być dwa zbiory
    takie, że jeden zawiera się w drugim.
\end{zad}
\begin{mdframed}
    \textit{Dowód indukcyjny}

    \begin{enumerate}
        \item Dla $n=2$ oczywiste
        \item Załóżmy, że jest to prawdą dla zbioru $n$ elementowego
              którego podzbiorów jest $2^n$. Podzbiorów zbioru $n+1$ elementowego
              jest $2^{n+1}$, $2^n$ bez elementu $n+1$
    \end{enumerate}
\end{mdframed}



\begin{zad}[BRAK]
    Dla $n$--elementowego zbioru $X$ rozważ pewną rodzinę jego podzbiorów
    $\mathcal{F}$, gdzie $|F| > n/2$ dla każdego $F \in \mathcal{F}$. Wykaż, że istnieje
    $x \in X$ należący do co najmniej połowy zbiorów z $\mathcal{F}$.
\end{zad}
\begin{mdframed}

\end{mdframed}


\begin{zad}[W]
    Dana jest kwadratowa szachownica $2n \times 2n$ z wyciętym jednym polem.
    Wykaż, że dla wszystkich wartości $n \geq 1$ możemy pokryć tę szachownicę kostkami w
    kształcie litery L (czyli kwadrat $2 \geq 2$ bez jednego pola).
\end{zad}
\begin{mdframed}
    Dowód indukcyjny.
    \begin{enumerate}
        \item $n = 1$; rysunek 1
        \item zał ind spełnione dla $n$. teza ind dla $n+1$. rysunek2 ogólnie dzielimy na
              cztery, zał ind, od pozostałych trzech odejmujemy po jednym kawałeczku, załatwiamy
              je z ind a kawałeczki załatwiamy jednym l-em
    \end{enumerate}
\end{mdframed}

\begin{zad}[W]
    Dana jest kwadratowa szachownica $n \times n$. Dla jakich wartosci $n\geq 1$
    możemy pokryć tę szachownicę kostkami wielkości $2 \times 2$ oraz $3 \times 3$.
\end{zad}
\begin{mdframed}
    Dla kostek 2x2 i 3x3 oczywiste. Dla ich wielokrotności oczywsite.

    Teraz łączenie. I go nie zrozumiałem. Well, shit happens.
    Ale do przemyślenia.
\end{mdframed}





\begin{zad}[BRAK]

\end{zad}
\begin{mdframed}
    zrobione, ale średnio z zapisem.
    Nie zrozumiałem, ale za to macie zdjątko tablicy.
    rysunek1.
\end{mdframed}



















\chapter{Zestaw 2}          % ZESTAW 2

\begin{zad}[R]
    Na ile sposobów można ustawić $n$ wież na szachownicy
    $n \times n$ tak, by żadne dwie nie znajdowały się w
    polu wzajemnego rażenia.
\end{zad}
\begin{mdframed}
    Starczy zauważyć, że dla każdej wieży wybieramy rząd i kolumnę
    w której się znajduje -- i tym samym zmniejsza liczbę dostępnych
    o jeden. Tak więc odpowiedź wynosi: \[n \cdot n \cdot (n-1) \cdot
        (n-1) \cdot ... \cdot 2 \cdot 2 \cdot 1 \cdot 1 =  n! \cdot n!\]
\end{mdframed}




\begin{zad}[R]
    Na ile sposobów można ustawić $k$ wież na szachownicy $n \times m$
    tak, by żadne dwie nie znajdowały się w polu wzajemnego rażenia.
\end{zad}
\begin{mdframed}
    Zadanie analogiczne od poprzedniego - z tym, że zmienił nam się
    rozmiar planszy, a ponadto nie wypełniamy jej całej. Zasada
    pozostaje jednak ta sama. Na start jednak warto założyć, że
    $k \leq max\{n, m\}$ (choć w sumie jeżeli tak nie jest, to
    odpowiedź to 0). Mając to już za sobą.
    \[n \cdot m \cdot (n-1) \cdot (m-1) \cdot ... \cdot (n - k +1) \cdot (m -k +1)\]
    (wykonujemy mnożenie $k + k$ elementów -- stąd to $-k + 1$).
\end{mdframed}




\begin{zad}[R]
    Znalźć definicje rekurencyjne następujących ciągów:
    \begin{enumerate}
        \item $a(n)$ -- liczba słów długości $n$ nad alfabetem
              $\{0, 1\}$, które nie zawierają dwóch jedynek koło siebie.
        \item  $b(n)$ -- liczba różnych pokryć prostokąta o wymiarze
              $2 \times n$ dominami wymiaru $2 \times 1$.
    \end{enumerate}
\end{zad}
\begin{mdframed}
    \begin{enumerate}
        \item Oczywiście $a(1)=2$, $a(2)=3$.  Rozważmy słowo $n$ elementowe.
              Zauważamy, że jeżeli ono kończy sie ono zerem to poprzedzające słowo $n-1$
              elementowe jest dowolne. Jeżeli natomiast kończy się jedynką,
              to poprzedzające słowo $n-2$ elementowe jest dowolne
              (tak jakby sie cofamy krok dalej by mieć dowolność).
              Stąd: $a(n)=a(n-1)+a(n-2)$.
        \item Analogicznie do poprzedniego. Jak wiemy $a(1) = 1$, $a(2) = 2$. Zastanówmy się nad $a(n)$:
              Rozważamy ciąg o długości $n$. Jeżeli na końcu jest blok poziomy,
              to wiemy że powstał on z ciągu długości $a(n-2)$.
              Jeżeli jest pionowy, to wiemy, że musiał on powstać z ciągu długości $n-1$.
              A stąd $a(n) = a(n-1) + a(n-2)$.
    \end{enumerate}
\end{mdframed}




\begin{zad}[W/R]
    Ile rozwiązań ma równanie $x_1 + x_2+x_3+x_4 = 7$:
    \begin{enumerate}
        \item gdzie $x_i$ są liczbami naturalnymi?
        \item gdzie $x_i$ są dodatnimi liczbami naturalnymi?
    \end{enumerate}
\end{zad}
\begin{mdframed}
    Starczy wykonać siatkę która na dole ma $x_1 ... x_4$, a w rzędach
    wartości od 0 (1) do 7. Wysokość nad $x_i$ oznacza wartość sumy do
    tego elementu włącznie, a przeskok względem poprzedniej wysokości
    oznacza wartość danego $x_i$. Więc można to potraktować jako ścieżki,
    a na mocy własności z wykładu wiemy ile jest ścieżek po kracie. Tak więc:
    \begin{enumerate}
        \item \[\binom{7+4-1}{7}=\binom{10}{7}\]
        \item \[\binom{3+4-1}{3}=\binom{6}{3}\]
    \end{enumerate}
\end{mdframed}




\begin{zad}[W]
    Rozważmy czekoladę złożoną z $m\times n$ kostek.
    Na ile sposobów można wykroić prostokąt złożony z $k \times k$
    sąsiadujących ze sobą kostek ze sobą kostek czekolady?
\end{zad}
\begin{mdframed}
    Na początek warto wykonać założenia $k \leq n$, $k \leq m$
    (albo stwierdzić że wtedy odpowiedź to zero). Potem zauważamy,
    że ilość sposobów można połączyć z ilością punktów startowych
    ulokowania prostokąta w tabliczce. Mamy ich
    \[(n-k+1)\times (m-k+1)\]
\end{mdframed}




\begin{zad}[W]
    (Reguła sumowania po górnym indeksie). Udowodnij, że dla
    $n, k \in \mathbb{N}$ zachodzi
    \[\sum_{j=0}^n\binom{j}{k} = \binom{n+1}{k+1}\]
\end{zad}
\begin{mdframed}
    Starczy zrobić indukcje. Wychodzi za darmo.
    Kombinarotrycznie też się da.
    Polecam spróbować samemu.
\end{mdframed}




\begin{zad}[W]
    (Reguła sumowania równoległego). Udowodnij, że dla $n, k \in \mathbb{N}$
    zachodzi \[     \]
\end{zad}\
\begin{mdframed}
    Starczy zrobić indukcje. Wychodzi za darmo.
\end{mdframed}




\begin{zad}[W]
    Ile jest funkcji $f:\{1, ..., n\} \to \{1, ..., n\}$ monotonicznych takich,
    że $f(i) \leq f(j) $ dla $i < j$?
\end{zad}
\begin{mdframed}
    Rozwiązanie korzystające z tej samej koncepcji co 4 z tego zestawu.
    Tworzymy kratkę, gdzie kolumny to argumenty, a rzędy to wartości.
    Interesują nas ścieżki z początku do końca - które generują nam wszystkie funkcje
    spełniające warunki zadania. Tak więc:
    \[ \binom{n + n -1 }{n} = \binom{2n-1}{n} \].
\end{mdframed}




\begin{zad}[R]
    Ile jest $k$--elementowych podzbiorów zbioru $n$--elementowego, które nie
    zawierają dwóch sąsiednich liczb?
\end{zad}
\begin{mdframed}
    Zauważamy, że możemy stworzyć funkcję `jest nie--jest'. Jest to funkcja charakterystyczna
    przyjmująca 0 lub 1 w zależności od tego czy element zbioru występuje, czy nie występuje w podzbiorze.
    Generujemy więc zbiór typu 10101 aż do uzyskania $k$ jedynek (tu $k=3$), ustalamy miejsca które możemy dopełniać zerami (by były to ciągi długości $n$ -- funkcja
    charakterystyczna). Czyli mamy $k-1$ zer na start i $n-(k)-(k-1)$ zer jeszcze do zrobienia.
    Czyli finalnie:
    \[\binom{n-k+1}{k}  \]
    I to było rozwiązanie pierwsze, myślę bardziej oczywiste rozwiązanie.Da się jednak prościej -- zaczynając od ciągu $n-k$ zer i wkładając jedynki w dozwolone miejsca.
\end{mdframed}


\begin{zad}[zawsze na kolokwim -- podwójne zliczanie]
    Posługując się interpretacją kombinatoryczną udowodnij, że:
    \[ \sum_{i=0}^{k} \binom{n}{i} \binom{n-i}{k-i} = 2^k \binom{n}{k} \]
\end{zad}
\begin{mdframed}
    %Zdjęcie 18.03.2025 16:36
    Historyjkowo: wybieramy z $n$ osób dwie drużyny, które w sumie mają mieć $k$ osób.

    L: wybieramy osoby do podziału na drużyny na $\binom{n}{k}$ sposobów.
    Każdą z $k$ osób które wybraliśmy wrzucamy albo do jednej albo do drugiej drużyny
    -- czyli mamy 2 możliwości. Stąd ilość podziałów na drużyny wynosi $2^k \binom{n}{k}$

    P: Ustalmy $i$ liczbę osób w pierwszej drużynie. Wybieramy jej skład -- robimy
    to na $\binom{n}{i}$ możliwości. Wybierzmy skład drugiej drużyny
    (z pozostałych nam nieprzydzielonych ludzi)
    -- składającej się z $k-i$ osób: $\binom{n-i}{k-i}$.
    Pierwsza drużyna może mieć od $0$ do $k$ osób, więc sumując po tych wartościach
    otrzymujemy: $\sum_{i=0}^{k} \binom{n}{i} \binom{n-i}{k-i}$.

    A stąd zauważamy, że L = P, czyli:
    \[ \sum_{i=0}^{k} \binom{n}{i} \binom{n-i}{k-i} = 2^k \binom{n}{k} \]


\end{mdframed}




\begin{zad}
    Udowodnij poniższe tożsamości na dwa sposoby: posługując się interpretacją
    kombinatoryczną albo rozwinięciem dwumianu $(1 + x)^n$:
    \begin{enumerate}
        \item \[\sum_{k=0}^{n}k\binom{n}{k} = n2^{n-1}\]
        \item \[\sum_{k=0}^{n}k^2\binom{n}{k}= (n+n^2)2^{n-2}\]
        \item \[\sum_{i=0}^{k}\binom{m}{i}\binom{n}{k-i} = \binom{m+n}{k} \]
    \end{enumerate}
\end{zad}
\begin{mdframed}
    \begin{enumerate}
        \item (można pochodną dwumianu) historyjka o wyborze króla i jego wojska
        \item (można drugą pochodną dwumianu)
    \end{enumerate}
\end{mdframed}

















\chapter{Zestaw 3}          % ZESTAW 3

\begin{zad}
    Wykaż, że dla dowolnego $n \geq 1$ istnieje $k \geq 1$ takie, że:
    \[S(n, 0) < S(n, 1) < ... < S(n, k - 1 ) \leq S(n, k) > S(n, k+1) > ... > S(n, n)\]
\end{zad}
\begin{mdframed}
    Zadanie trudne, na ćwiczeniach prowadzący stwierdził, że nie robimy -- bo za trudne.
    Powodzenia!
\end{mdframed}




\begin{zad}
    Wykaż, że:
    \[B(n) = \sum_{i=0}^{n-1} \binom{n-1}{i}B(i)\]
\end{zad}
\begin{mdframed}
    Historyjkowo:

    Weźmy zbiór $n-1$ elementowy. Wybierzmy z niego $k$ elementów, co
    możemy zrobić na $\binom{n-1}{k}$ sposobów. Następnie dołączamy do
    wybranych elementów $n$ element. Pozostałe $n-k-1$ elementów
    możemy podzielić na $B(n-k-1)$ sposobów. Sumując k od $0$ do $n-1$:
    \[B(n) = \sum_{k=0}^{n-1}B_{(n-k-1)}\binom{n-1}{k} \overset{i=n-1-k}{=}
        \sum_{i=0}^{n-1}B(i)\binom{n-1}{n-1-i} \]
    \[B(n) = \sum_{i=0}^{n-1} \binom{n-1}{i}B(i)\]



\end{mdframed}




\begin{zad}
    Wykaż, że dla $n, k \in \mathbb{N}$ zachodzi:
    \[S(n,k+1)=\frac{1}{(k+1)!} \sum_{0<i_0<...<i_{k-1}<n} \binom{n}{i_{k-1}}\binom{i_{k-1}}{i_{k_2}}...\binom{i_1}{i_0}     \]
\end{zad}
\begin{mdframed}
    Historyjkowo:

    Zacznijmy od przepisania powyższego równania do trochę innej formy:
    \[(k+1)! \cdot S(n,k+1) = \sum_{0<i_0<...<i_{k-1}<n} \binom{n}{i_{k-1}}\binom{i_{k-1}}{i_{k_2}}...\binom{i_1}{i_0}     \]

    Zastanówmy się co mamy po prawej. Mamy sumę po wszystkich monotonicznych ciągach od $0$ do $n$.
    Weźmy jeden konkretny. Weźmy ponadto jakiś zbiór `duży' (nie w sensie wielkości, ale konceptu) $n$ elementowy
    zbiór i weźmy wybierzmy z niego $i_{k-1}$ elementów (robimy to na $\binom{n}{i_{k-1}}$ -- sposobów).
    Następnie z tego naszego zbioru $i_{k-1}$ elementów wybierzmy zbiór $i_{k-2}$ elementowy
    (robimy to ponownie na $\binom{i_{k-1}}{i_{k-2}}$); i tak dalej po wszystkich wyrazach naszego ciągu.
    I co otrzymujemy? Podział naszego `dużego' zbioru $n$ -- elementowego na $k+1$ zbiorów, ale z kolejnością.
    A co mamy po lewej? Podział zbioru $n$ elementowego na $k+1$ podzbiorów -- ale bez kolejności, którą można
    dorzucić za pomocą $(k+1)!$.
    \[
        L=P
    \]


\end{mdframed}





\begin{zad}
    Rozważ  następującą procedurę generującą pewne liczby naturalne
    $\{a_{i,j}\}_{1 \geq i \geq j}$:
    \begin{enumerate}
        \item $a_{0,0} = 1$,
        \item $a_{n+1, 0} = a_{n,n}$, dla $n \geq 0$,
        \item $a_{n+1, k+1} = a_{n, k} + a_{n+1, k}$, dla $n \geq k \geq 0$.
    \end{enumerate}
\end{zad}
\begin{mdframed}
    Zadanie trudne, niestety sam nie zrobiłem ani nie zapisałem rozwiązania na ćwiczeniach.
    Wiem, że w rozwiązaniu wykorzystaliśmy pracę \href{https://www.sciencedirect.com/science/article/pii/S0195669810001502}{dostępną tutaj!}.
\end{mdframed}




\begin{zad}
    Wykaż, że liczba podziałów zbioru $(n - 1)$  elementowego jest równa
    liczbie podziałów zbioru $\{1, ..., n\}$ niezawierających sąsiednich liczb w jednym bloku.
\end{zad}
\begin{mdframed}
    Rozważmy funkcję
    pomiędzy podziałem zbioru $n-1$ elementowego
    a podziałem zbioru $\{1, ..., n\}$:

    Po pierwsze, poindeksujmy po kolei elementy naszego podziału $[n-1]$.
    Idąc od prawej po naszym $n-1$ elementowym podziale wybieramy z każdego bloku co drugą liczbę zaczynając od drugiej,
    wyciągamy ją i idziemy dalej. Po przejściu przez cały podział dorzucamy do niego nowy zbiór:
    składający się z wszystkich wyciągniętych przez nas elementów i $n$.

    Na przykład:
    \[ 12|3|45678 \to 864|3|2|7519\]

    \begin{center}
        \textit{(trzeba jeszcze tylko udowodnić, że ta funkcja jest bijekcją)}
    \end{center}

    Zauważamy, że w ten sposób jednoznacznie stworzyliśmy zbiór będący podziałem zbioru $\{1, ..., n\}$.
    A jak wiemy z \textit{ELiTMu} jeżeli pomiędzy zbiorami istnieje bijekcja to są równoliczne.
\end{mdframed}




\begin{zad}
    Udowodnij, że liczba ukorzenionych drzew binarnych na $n$ wierzchołkach to $n$-ta liczba Catalana.

    Ukorzenione drzewo jest drzewem binarnym, jeśli każdy wierzchołek ma co najwyżej
    dwójkę dzieci przy czym co najwyżej jedno lewe dziecko i co najwyżej jedno prawe dziecko.
\end{zad}
\begin{mdframed}
    Chcemy dowieźć, że liczba ukorzenionych drzew binarnych $N(n)$ jest równa $B(n)$.

    Na start zauważamy, że $F(1) = 1 = B(1)$.

    Teraz chcemy pokazać, że: $F(n) = \sum_{i=0}^{n-1}F(i)F(n-i-1)$.
    Rozważmy drzewo $n$ wierzchołkowe.
    Niech prawe dziecko ma korzeń z $i$ wierzchołkami a lewe z $n - i - 1$.
    Czyli istnieje $F(i)$ możliwych drzew z dla prawego dziecka i $F(n-1-i)$ dla lewego.
    Czyli dla tego zespołu istnieje $F(i) \cdot F(n-1-i)$ możliwych drzew.
    Wiemy, że liczba $i$ wierzchołków po prawej może osiągnąć wartości od $0$
    do $n-1$ czyli
    \[F(n) = \sum_{i=0}^{n-1}F(i)F(n-i-1)          \]

    Co jest rekurencją spełnianą przez liczby Bella.
\end{mdframed}




\begin{zad}
    Triangulacją $n$ -- wierzchołkowego wielokąta wypukłego nazywamy zbiór
    $(n - 3)$ wzajemnie nieprzecinających się jego przekątnych, które dzielą jego obszar na
    $(n - 2)$ trójkątów.
    \begin{enumerate}
        \item ile jest triangulacji $n$--wierzchołkowego wielokąta wypukłego?
        \item Ile jest triangulacji $n$--wierzchołkowego wielokąta wypukłego, w których każdy trójkąt
              triangulacji ma przynajmniej jeden bok na brzegu wielokąta?
    \end{enumerate}
\end{zad}
\begin{mdframed}
    Polecam wziąć sobie kartkę i rysować, bo może być ciężko.
    \begin{enumerate*}
        \item Weźmy $n+2$ wierzchołkowy wielokąt wypukły. Zastanówmy się ile jest triangulacji $T(n)$?
              Weźmy jeden z jego wierzchołków. Wybierzmy ponadto po kolei kolejne krawędzie
              tego wielokąta i utwórzmy z na nich trójkąty.
              Zauważamy, że możemy wybrać $n$ krawędzi, stworzyć na nich trójkąty, a dla powstałych
              wielokątów zapisać rekurencję. Tak więc wynikowa suma po $i \in [n]$ krawędziach będzie sumą po wszystkich podziałach
              utworzonych po wszystkich trójkątach, tak więc:
              \[T(n) = \underbrace{\underbrace{T(0)}_{lewa} \cdot \underbrace{T(n-1)}_{prawa}}_{\text{dla trójkąta na $i = 1$}}  +  T(1) \cdot T(n-2) \cdot ... \cdot T(n-1) \cdot T(0)\]

              Co jest rekurencją spełnianą przez liczby Catalana.

        \item Na pewno jest jakieś elegantsze rozwiązanie, ale da się stwierdzić, że liczba ta wynosi

    \end{enumerate*}



\end{mdframed}
\begin{mdframed}
    \begin{enumerate}
        \item
              Weźmy bok 12 -- jest on częścią pewnego trójkąta 12 i (po triangulacji)
              Jeśli $T_{n+2}$ = wszystkie figury
              $T_{n+2}$ = te triangulacje z $T_{n+2}$, które będą zwierały $\Delta12$
              i stąd ${T_{n+2}^i}$ sa parami rozłączne oraz
              $T_{n+2} = U_{i+3}^{n+1}T_{n+2}^i$
              Mamy $|T_{n+2}| = \sum_{i = 3}^{n+1}|T_{n+2}^i| = \sum_{i = 3 }^{n+1}|T_i| \cdot |T_{n+3-i}|$
              Niech $|T_2| = 1$ i $|T_3| = 1$.

              Co po przekształceniach da $n$--tą liczbę Catalana


        \item
              Podpunkt 2 -- dokładnie tak samo jak 1.
              Niech figura ma $n$ wierzchołków.
              Wybieramy bok 12 oraz zauważamy że 12 jest częścią pewnego trójkąta $\Delta12$
              i:
              \begin{enumerate}
                  \item
                        $i = 3$, $n$
                        Zauważmy, że w przypadku i = 3 krawedz 13 musi byc
                        czescia trojkata $\Delta134$ lub $\Delta13n$ i bla bla bla ostatecznie generujemy procedure gdzie tak schodkowo generujemy, i dla kazdej krawedzi mamy dwie możliwości zbudowania nowego trojakąta.
                        Ile zatem takich możliwości jest? Jest ich $2^{n-4}$.
                        Sytuacja symetryczna dla $n$.
                  \item $i \neq 3$ oraz $i \neq n$
                        \[2^{i-4} \cdot 2^{n-i-1} = 2^{n-5} \]

              \end{enumerate}
              Czyli sumarycznie:
              \[ 2^{i-4} + 2^{i-4} + (n-4)\cdot 2^{n-5}\]
              czyli do $(n\cdot 2^{n-5})$

    \end{enumerate}

\end{mdframed}




\begin{zad}[R/prywata]
    Wykaż, że liczba drzew etykietowanych na zbiorze ${1, ..., n}$ wynosi $n^{n-2}$.
\end{zad}
\begin{mdframed}
    Drogi studencie, to jest jedno z tych zj@b@nych zadań, które moim zdaniem
    nie powinno sie tu znaleźć.

    Jest to nietrywialny problem świata rzeczywistego -- który całe szczęście
    ma swoje rozwiązanie (o grozo z nazwiskiem!).

    Algorytm, będący bijekcją pomiędzy dowolnym (a de facto większym od 2) grafem
    a ciągiem $n-2$ liczb o możliwych wartościach $[n]$ -- którego liczność to oczywiście
    \[n^{n-2}\]
    \href{https://pl.wikipedia.org/wiki/Kod_Pr%C3%BCfera}{znajdziesz tutaj!}
    Jedynym problem pozostaje udowodnienie, że powyższa funkcja faktycznie jest bijekcją...
\end{mdframed}






\chapter{Zestaw 4}          % ZESTAW 4

\begin{zad}
    Oblicz $S(n, 2)$.
\end{zad}
\begin{mdframed}
    Czyli \textit{de facto} pytanie o to, na ile sposobów można podzielić
    na 2 niepuste zbiory zbiór $n$ elementowy.
    \[S(n, 2) = \frac{\overbrace{2^n}^{\text{wszystkie podzbiory}} - \overbrace{2}^{\text{bez pustego i całego}}}{\underbrace{2}_{\text{nie interesuje nas kolejność}}} = 2^{n-1}-1\]

    \textit{Oczywiście da się to zrobić też rekurencyjnie -- choć jest to dłuższe zadanie (które mimo wszystko polecam zrobić).}
\end{mdframed}




\begin{zad}
    Wykaż, że mamy dokładnie
    \[\frac{n!}{1^{\lambda_1} \cdot 2^{\lambda_2} \cdot  ... \cdot n^{\lambda_n} \cdot \lambda_1! \cdot ... \cdot \lambda_n!}\]
    permutacji zbioru $[n]$ o typie $1^{\lambda_1} \cdot 2^{\lambda_2} \cdot ... \cdot n^{\lambda_n} $  (mających $\lambda_i$ cykli długości i dla $i \in [n]$).
\end{zad}
\begin{mdframed}
    Permutację $[n]$ jako funkcję można utożsamiać z ciagiem $n$ elementowym,
    przy czym ważne jest to, że nie istotny jest początek ani koniec tego ciągu
    (bo permutacja to cykl).

    Zauważamy, że wszystkich permutacji $[n]$ jest $n!$.
    Zapiszmy te cykle w postaci ciągu:
    \[\underbrace{(\cdot)(\cdot) ... (\cdot)}_{\lambda_1}\underbrace{(\cdot \cdot)...(\cdot \cdot)}_{\lambda_2} ... \underbrace{(\underbrace{\cdot ... \cdot}_{n})...(\underbrace{\cdot ... \cdot}_{n})}_{\lambda_n}\]

    Usuńmy z każdego $i$ elementowego ciągu kolejność
    (bo chcemy cykle, nie stricte ciągi), czyli dla każdego podzielmy przez $i$,
    co robimy finalnie dzieląc przez $i^{\lambda_i}$ dla każdego $i$.

    Ponadto, zauważamy, że nasz sposób ułożenia cykli zakłada kolejność,
    której pozbędziemy się dzieląc przez $\lambda_i!$ dla każdej grupy cykli.

    Czyli finalnie liczba takich cykli będzie równa:
    \[\frac{n!}{1^{\lambda_1} \cdot 2^{\lambda_2} \cdot  ... \cdot n^{\lambda_n} \cdot \lambda_1! \cdot ... \cdot \lambda_n!}\]

    %stare, zle mazulisa
    %  Dla każdej $\lambda_i$ jest $i$ cykli w nawiasie, więc będziemy dzielić przez:
    % \[1^{\lambda_1}2^{\lambda_2}\cdot ... \cdot n^{\lambda_n}\]
    %Wewnątrz każdego cyklu jest $\lambda_i!$ permutacji, więc dzielmy przez:
    %\[(\lambda_1! \lambda_2! \cdot ... \cdot \lambda_n!)\]
    %nowe, dobre, jawora
\end{mdframed}




\begin{zad}
    Posługując się interpretacją kombinatoryczną, wykaż tożsamość:
    \[S(n+1,m+1) = \sum_k \binom{n}{k}S(k,m)\]
\end{zad}
\begin{mdframed}
    Kombinatorycznie:

    \textit{Lewa:}
    Liczba podziałów zbiory $\left[n+1\right]$ na $m+1$ bloków:
    \[S(n+1,m+1)\]

    \textit{Prawa:}
    Wybieramy $k$ liczb które nie będą w nowym bloku z liczbą $n+1$.
    Pozostałe będą z $n+1$. Spośród liczb wybranych tworzymy $m$ bloków
    na $S(k,m)$ sposobów. Sumując po wszystkich możliwych $k$:
    \[\sum_k \binom{n}{k}S(k,m)\]

    \begin{center}
        L=P
    \end{center}

\end{mdframed}




\begin{zad}
    Zakładając, że zachodzi równość:
    \[
        (x_1 + ... + x_k)^n = \sum_{n_1+...+n_k=n}\binom{n}{n_1 n_2 ... n_k}x_1^{n_1}\cdot...\cdot x_k^{n_k}
    \]
    podaj ile wynosi $\binom{n}{n_1 n_2 ... n_k}$.
\end{zad}
\begin{mdframed}
Zadanie koncepcyjne w zasadzie analogiczne do kombinatorycznego wyprowadzenia
$(x+y)^n$.

Iloczyn $x_1^{n_1}\cdot ... \cdot x_k^{n_k}$ powstaje w sposób taki, że z każdego z nawiasów:
\[(x_1 + ... + x_k) \cdot (x_1 + ... + x_k) \cdot ... \cdot (x_1 + ... + x_k)\]
wybieramy jedną z liczb od $x_1$ do $x_k$, przy czym $x_i$ wybieramy
$n_i$ razy.
Wybierzmy z których nawiasów wybieramy. Dla $n_1$ wybieramy z $\binom{n}{n_1}$
nawiasów, dla $n_2$ z $\binom{n-n_1}{n_2}$ itd. dla $n_k$ z
$\binom{n-n_1-n_2-...n_{k-1}}{n_k} = \binom{n_k}{n_k}$.
I robimy tak po wszystkich możliwych układach, ale wiemy, że lewa
strona założenia równa się prawej, więc:
\[
    \binom{n}{n_1 n_2 ... n_k} = \binom{n}{n_1} \binom{n-n_1}{n_2} ... \binom{n_k}{n_k}
\]

\begin{center}\href{https://lidicky.name/oldteaching/20.569X/l03%20-%20Binomial%20Theorem.pdf}{(można przeczytać dodatkowo)}\end{center}
\end{mdframed}




\begin{zad}
    Wykaż, że
    \[
        \sum_{i=0}^{n} i \left[ n \atop i \right] = n! H_n,
    \]
    gdzie $H_n = 1 + \frac{1}{2} + ... + \frac{1}{n}$.
\end{zad}
\begin{mdframed}
    Historyjkowo:

    \textit{Lewa:}
    Zastanówmy się, ile jest permutacji $[n]$ w których wyróżniamy jeden cykl.
    Ustalmy $i$. Generujemy wszystkie permutacje $n$-elementowe o $i$ cyklach za pomocą
    $\left[ n \atop i \right]$ i wybieramy z niej wyróżnioną na $i$ sposobów.
    Sumując po wszystkich $i$ (liczbach cykli) otrzymujemy:
    \[\sum_{i=0}^{n} i \left[ n \atop i \right]\]

    \textit{Prawa:}
    Zróbmy to samo, ale inaczej. Ustalmy $k$ -- liczbę elementów
    w wyróżnionym cyklu. Stwórzmy wyróżniony cykl,
    możemy zrobić to na $\binom{n}{k} \cdot i! \cdot \frac{1}{k}$.
    Z reszty elementów tworzymy permutację na $(n-k)!$ sposobów.
    Czyli finalnie otrzymujemy:
    \[
        \sum_{k=1}^{n}\binom{n}{k}\frac{k!}{k}(n-k)! = \sum_{k=1}^{n}\frac{n!}{k} = n! \cdot H_n
    \]
\end{mdframed}
\begin{mdframed}
    $X$ -- permutaje zbioru $n$--elementowego z jednym wyróżnioneym cyklem:
    \[\left\{(\pi, c):  \right\}\]

\end{mdframed}




\begin{zad}[]
    Wykaż, że dla dowolnego $x \in \mathbb{R}$ zachodzi:
    \begin{enumerate}
        \item $x^n = \sum_{k}S(n,k) x^{\underline{k}}$
        \item $x^{\bar{n}} = \sum_{k} \left[n \atop k \right] x^k$.
    \end{enumerate}
\end{zad}
\begin{mdframed}
    \begin{center} Udowodnimy najpierw dla $x \in \mathbb{N}$, potem zbiorczo uzasadnimy dla $\mathbb{R}$.\end{center}
    \begin{enumerate*}
        \item Historyjkowo:

              \textit{Lewa:}
              Wrzucamy $n$ osób do $x$ pokoi. Robimy to na $x^n$ sposobów.

              \textit{Prawa:}
              Dzielimy $n$ osób na $k$ grup -- robimy to na $S(n,k)$ sposobów.
              Wybieramy z wszystkich $x$ pokoi $k$ pokoi na $\binom{x}{k}$
              sposobów i przydzielamy je do grup na $k!$ sposobów.
              Tak więc sumarycznie sumując po wszystkich licznościach $k$:
              \[\sum_{k} S(n, k) \cdot \binom{x}{k} \cdot k! = \sum_{k} S(n, k) \cdot x^{\underline{k}} \]

              Pozostaje jedynie kwestia uzasadnienia, że dzieje się tak
              nie tylko dla $x \in \mathbb{N}$, ale i dla $x \in \mathbb{R}$. \newline


        \item Historyjkowo:

              \textit{Prawa:}
              Będziemy permutować $[n]$, a powstałe cykle dzielić pomiędzy szuflady.
              Ustalmy $k$. Perutmujemy zbiór $[n]$ na $k$ cykli, robimy to na $\left[n \atop k \right]$ sposobów.
              Każdy z otrzymanych cykli wrzucamy do szufladki, robimy to na $x^k$ sposobów.

              Sumując po wszystkich $k$ otrzymujemy:
              \[\sum_k \left[n \atop k \right] \cdot x^k\]

              \textit{Lewa:}
              Bierzemy po kolei elementy z $[n]$ i wrzucamy je do dostępnych miejsc (szufladek i fragmentów permutacji).
              Pierwszy możemy wrzucić na $x$ sposobów, drugi na $(x+1)$, trzeci na $(x+2)$ itd. Sumarycznie, dla wszystkich elementów:
              \[ \underbrace{x \cdot (x+1) \cdot (x+2) \cdot ... \cdot (x + n - 1)}_{\text{$n$ elementów}} = x^{\underbar{n}} \] \newline

              Udowodniliśmy powyższe równości dla $x \in \mathbb{N}$, zauważmy, że po obu stronach
              równości mamy wielomiany, przerzucając je na jedną stronę zauważymy, że powstały
              wielomian ma nieskończenie wiele miejsc zerowych, jedyny taki wielomian to wielomian
              zerowy, wielomiany po obu stronach równości są sobie równe, więc równość zachodzi także dla $x \in \mathbb{R}$.
    \end{enumerate*}
\end{mdframed}














\chapter{Zestaw 5}          % ZESTAW 5

\begin{zad}
    Wykaż zasadę włączeń i wyłączeń korzystając z indukcji po liczbie zbiorów.
\end{zad}
\begin{mdframed}

\end{mdframed}




\begin{zad}[Autorem rozwiązania jest: `Kuba Jaworski']
    Wykaż, że mamy
    \[
        \sum_{j=0}^{m}(-1)^j \binom{m}{j}(m-j)^n
    \]
    suriekcji ze zbioru $[n]$ w zbiór $[m]$.
\end{zad}
\begin{mdframed}
    Funkcja będąca surjekcją jest funkcją, której zbiór wartości jest
    równy przeciwdziedzinie.

    Zauważmy, że liczba surjekcji to liczba wszystkich funkcji odjąć
    liczba funkcji nie będących surjekcjami.

    Wszystkich funkcji $f:[n] \to [m]$ mamy: $n^m$.
\end{mdframed}



\begin{zad}
    Ile jest ciągów długości $2n$ takich, że każda liczba $i \in [n]$
    występuje dokładnie dwa razy oraz kaze sąsiednie dwa wyrazy są różne.
\end{zad}
\begin{mdframed}

\end{mdframed}




\begin{zad}
    Wykaż, że dla $n \geq 3$ zachodzi tożsamość
    \[
        D(n) = (n-1)(D(n-1) + D(n-2))
    \]
    gdzie $D(n)$ jest liczbą permutacji zboru $[n]$ bez punktów stałych.
\end{zad}
\begin{mdframed}

\end{mdframed}




\begin{zad}
    Wykaż (najlepiej kombinatorycznie), że dla dowolnych $n, k \in \mathbb{N}$
    zachodzi:
    \begin{enumerate}
        \item $S(n, k) = \sum_{0 \leq m_1 \leq m_2 \leq ... \leq m_{n-k} \leq k} m_1m_2 \cdot ... \cdot m_{n-k}$
        \item $c(n, k) = \sum_{0 < m_1 < m_2 < ... < m_{n-k} < k} m_1m_2 \cdot ... \cdot m_{n-k}$
    \end{enumerate}
\end{zad}
\begin{enumerate}
    \item z1
    \item z2
\end{enumerate}



\begin{zad}
    Ciąg podziałów zbioru ${1, ..., n}$ tworzymy następująco. Startujemy
    od podziału zawierającego tylko zbiór ${1, ..., n}$. Podział $(i + 1)$--wszy
    otrzymujemy z podziału $i$-tego poprzez:
    \begin{enumerate}
        \item wybranie jednego, co najmniej $2$-elementowego zbioru z podziału $i$-tego i podzielenie
              go na dwa niepuste podzbiory,
        \item podzielenie każdego, co najmniej $2$-elementowego zbioru z podziału $i$-tego na dwa
              niepuste podzbiory.
    \end{enumerate}
    W obu przypadkach procedura kończy swoje działanie jeżeli wszystkie zbiory podziału są
    jednoelementowe. Na ile sposobów można wykonać powyższe procedury? Na ile sposobów
    możemy wykonać powyższe procedury zakładając, że po każdym kroku zbiory podziałów
    zawierają kolejne liczby naturalne?
\end{zad}
\begin{mdframed}

\end{mdframed}



\end{document}