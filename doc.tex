\documentclass[12pt]{article}
\usepackage[utf8]{inputenc}
\usepackage[a4paper, total={16cm, 25cm}]{geometry}
\usepackage{polski}
\usepackage{amsmath}
\usepackage{amssymb}
% \usepackage{amsfonts}
\usepackage{enumitem}
\usepackage{amsthm}
\usepackage{cancel}
\usepackage{tikz}
\usepackage{cases}
\usepackage{listings}
\usepackage{array}
\usepackage{fancyhdr}
\usepackage{stackrel}
\usepackage{hyperref}
\usepackage{mathtools}
\usepackage{tikz-among-us}
\usepackage{tikz-cd}
\usepackage{tocloft}
\usepackage{wasysym}
\usepackage{upgreek}
\usepackage{mdframed}
\usepackage{eso-pic}
% \usepackage[cor=pink!80!white,BG,type=0]{tikz-among-us-watermark-eso-pic}
\newtheorem{definition}{Definicja}
\newcommand{\R}{\mathbb{R}}
\newcommand{\N}{\mathbb{N}}
\newcommand{\Z}{\mathbb{Z}}
\newcommand{\C}{\mathbb{C}}
\mdfsetup{%
% frametitlerule=true,
% subtitlebelowline=true,
subtitleaboveline=true,
subtitleaboveskip=0,
subtitlebelowskip=0,
linewidth=1pt}
\usetikzlibrary{angles,quotes,calc}
\usetikzlibrary{arrows.meta}
\newcommand{\fl}{\operatorname{fl}}
\newcommand{\tr}{\operatorname{tr}}
\newcommand{\cond}{\operatorname{cond}}
\newcommand{\aar}[1][]{\ar[#1, shorten=2mm, end anchor=west, start anchor=east]}
\newenvironment{lista}[2][]{\newpage
\begin{center}
    \section[Lista \textbf{#2}]{Metody Numeryczne 1, lista #2}
\end{center}
\bigskip
#1
\begin{enumerate}[label=\textbf{#2\arabic*.}]}{\end{enumerate}}
\newcommand{\zad}[1][0.5]{\item \emph{(#1 pkt.) }}
\newcommand{\linia}{\centerline{\textcolor{gray}{\rule{10cm}{0.4pt}}}}
\renewcommand{\cftsecleader}{\cftdotfill{\cftdotsep}}
\renewcommand{\cftsecafterpnum}{\hspace*{7.5em}}
\setlength{\cftsecindent}{7.5em}

\author{\textbf{Maksymilian Soszyński} \\ \textsf{Informatyka i Systemy Informacyjne | MiNI PW}}
\title{%
%\includegraphics[width=0.8\textwidth]{wmini.png}~\\
\textbf{Metody Numeryczne I}}
\date{}
\setcounter{secnumdepth}{0}
\pagestyle{fancy}
\setlength{\headsep}{1.2cm}
\fancyhead{}
\lhead{Maksymilian Soszyński\\\textsf{Wydział MiNI PW}}
\chead{\large \textbf{Rozwiązania zadań}}
\rhead{\textbf{Metody Numeryczne I}\\\nouppercase\leftmark}

\fancyfoot{}
\rfoot{\texttt{\thepage}}
% \rfoot{
%     \begin{tikzpicture}
%     \amongUsI[rotate around={-45*(\thepage-2):(1.75,2.3)},
%     scale=0.25, shift={(5,7)}]{red}{cyan}
%     \node at (1.75,2.3) {\bfseries\thepage};
%     \end{tikzpicture}
% }
\fancypagestyle{specialfooter}{%
    \fancyhf{}
    \renewcommand\headrulewidth{0pt}
    \fancyfoot[C]{\date{\today}}
}

\begin{document}
\maketitle
\begin{center}
    Rozwiązania zadań w roku akad. 2022/23 (sem.\ letni). \\
    Ćwiczenia z \emph{numerków} prowadziła dr inż. Iwona Wróbel. \\
    (Mam nadzieję, że błędów jest jak najmniej, starałem się.) \\
    \tableofcontents
    % \thispagestyle{empty}
    \thispagestyle{specialfooter}

\end{center}

\begin{lista}[\paragraph*{Definicja.} Macierz $A\in\R^{n\times n}$ jest \emph{dodatnio określona}, jeśli $x^TAx>0$ dla każdego niezerowego wektora $x\in\R^n$.]{A}
    \zad Pokaż, że jeśli $A\in\R^{n\times n}$ i $B\in\R^{n\times n}$ są dodatnio określone, to $A + B$ też jest dodatnio określona.

    Czy iloczyn dwóch macierzy dodatnio określonych jest macierzą dodatnio określoną? Odpowiedź uzasadnij.
    \begin{mdframed}
        \begin{proof}
            Niech $A$ i $B$ będą dodatnio określone. Z definicji mamy
            \[ \forall_{0\neq x\in\R^n} \quad x^TAx > 0 \land x^TBx > 0. \]
            Dodając nierówności stronami otrzymujemy
            \[ \forall_{0\neq x\in\R^n} \quad x^TAx + x^TBx > 0 \]
            Wyłączamy $x^T$ przed nawias oraz $x$ za nawias, korzystając z rozdzielności
            \[ \forall_{0\neq x\in\R^n} \quad x^T(A+B)x > 0. \qedhere \]
        \end{proof}

        \mdfsubtitle{Kontrprzykład.
            $A = \begin{bmatrix} 1 & -1 \\ 1 & 1 \end{bmatrix}$, \quad $B = \begin{bmatrix} 1 & -2 \\ 2 & 1 \end{bmatrix}$}

        Macierze A i B są dodatnio określone ($a,b\in\R, \;a\neq0\lor b\neq0$):
        \[ [a\ b] A \begin{bmatrix} a \\ b \end{bmatrix} = a^2+b^2>0 \quad\text{oraz}\quad[a\ b] B \begin{bmatrix} a \\ b \end{bmatrix} = a^2+b^2>0.\]
        Ale wtedy macierz $AB = \begin{bmatrix} -1 & -3 \\ 3 & -1 \end{bmatrix}$ oraz
        \[[a\ b] AB \begin{bmatrix} a \\ b \end{bmatrix} = -a^2-b^2 < 0 \quad \text{(np.} \; [a\ b] = [1\ 1]\text{).} \]
        Czyli iloczyn dwóch macierzy dodatnio określonych nie musi taki być.\\
        \emph{W późniejszej części semestru zostanie wprowadzone Kryterium Sylvestra, które pozwoli badać określoność wyznacznikami danych macierzy symetrycznych.}
    \end{mdframed}

    \zad[0.75] Pokaż, że wszystkie wartości własne macierzy symetrycznej dodatnio określonej są dodatnie.

    Czy macierz, której wszystkie wartości własne są dodatnie, jest dodatnio określona? Odpowiedź uzasadnij.

    \begin{mdframed}
        \begin{proof}
            Niech $A$ będzie macierzą symetryczną dodatnio określoną. Macierz $A$ jest symetryczna, więc wszystkie wartości własne macierzy są rzeczywiste. \\ Niech $\lambda$ będzie wartością własną macierzy dla niezerowego wektora $x\in\R^n$.

            Wtedy
            \[ x^T(Ax) = x^T(\lambda x) = \lambda(x^Tx) = \lambda \sum_{i=1}^n x_i^2 > 0. \]
            Z założenia dodatniej określoności wiemy, że ten iloczyn jest dodatni, a suma kwadratów elementów wektora jest dodatnia. Wartość własna $\lambda$ jest więc dodatnia. \qedhere \\
        \end{proof}

        \mdfsubtitle{Kontrprzykład. $A = \begin{bmatrix} 1 & 0 \\ 10 & 1 \end{bmatrix}$}

        \[ \det(A-\lambda I) = \det \begin{bmatrix} 1 - \lambda & 0 \\ 10 & 1 - \lambda \end{bmatrix} = (1-\lambda)^2 = 0 \]
        Wartością własną macierzy A jest $\lambda = 1 > 0$, ale dla wektora $x = \begin{bmatrix} -1 \\ 1 \end{bmatrix}$ mamy
        \[ x^TAx = -8 \leq 0. \]
        Więc macierz $A$ nie jest dodatnio określona.
    \end{mdframed}

    \zad\label{A3} Wykaż, że jeśli $A\in\R^{n\times n}$ jest dodatnio określona, to $a_{i,i} > 0$ dla każdego $i = 1,\dots,n$ ($a_{i,i}$ oznacza element leżący na głównej przekątnej macierzy $A$, w $i$-tym wieszu).

    \begin{mdframed}
        \begin{proof}
            Niech $A$ będzie macierzą dodatnio określoną. Niech $x_i = \begin{bmatrix} 0 \\ \vdots \\ 1 \\ \vdots \\ 0 \end{bmatrix}$, gdzie $1$ jest w $i$-tym wierszu. Wtedy
            \[ x_i^T\overbrace{\begin{bmatrix} a_{1,1} & \cdots & a_{1,n} \\ \vdots & \ddots & \vdots \\ a_{n,1} & \cdots & a_{n,n} \end{bmatrix}}^A x_i = a_{i,i}\]
            $x_i^TA$ --- $i$-ty wiersz\\
            $x_i^TAx_i$ --- $i$-ty element $i$-tego wiersza.

            Ale z założenia $\forall_{0\neq x\in\R^n} \quad x^TAx > 0$ otrzymujemy $\forall_{i \in\{1,\dots,n\}}\quad a_{i,i} > 0$.
        \end{proof}
    \end{mdframed}
\end{lista}

\begin{lista}{B}
    \zad Zbadaj uwarunkowanie zadania obliczania wartości ekstremalnej paraboli $y(x) = x^2 + bx + 1$ w zależności od parametru $b$.

    \begin{mdframed}
        Niech $q$ będzie wartością ekstremalną tej paraboli. Wtedy
        \[ q(b) = \frac{-\Delta}{4a} = \frac{-(b^2-4)}{4} = \frac{-b^2+4}{4} = -\frac{1}{4}b^2 + 1 \]
        \[ C(x) = \biggl|\frac{b\cdot q'(b)}{q(b)}\biggr| = \biggl|\frac{-2b^2}{-b^2+4}\biggr| = \biggl|\frac{2b^2}{b^2-4}\biggr| \]
        Dla $b \approx 2$ i $b \approx -2$ wartość $C(b)$ jest bardzo duża (zadanie jest źle uwarunkowane).\\
        Dla $b \approx 0$ wartość $C(b)$ jest bliska $0$, błędy zostaną stłumione.
        \[ \lim_{x\to\pm\infty} C(x) = 2\]
        Dla $|x - 2| \gg 0$ i $|x + 2| \gg 0$ wartość $C(x)$ jest bliska $2$ (zadanie jest dobrze uwarunkowane).
    \end{mdframed}

    \zad Zbadaj uwarunkowanie zadania obliczania pola trójkąta równoramiennego o podstawie $2$ i kącie $\alpha$ przy podstawie $(0 < \alpha < \pi/2)$ w zależności od $\alpha$.

    \begin{mdframed}
        Ze wzoru na pole i definicji tangensa, gdzie a -- krawędź boczna, h -- wysokość, P -- pole:
        \[ P = \frac{1}{2}\cdot2\cdot h \Rightarrow P = h \]
        \[ \tg \alpha = \frac{h}{1} \Rightarrow h = \tg \alpha \]
        \[ P = \tg \alpha \]
        \[ C(\alpha) = \biggl|\frac{\alpha\cdot P'}{P}\biggr| = \biggl|\frac{\alpha\cdot \tg \alpha'}{\tg \alpha}\biggr| = \biggl|\frac{2\alpha}{\sin(2\alpha)}\biggr| \]
        Dla $\alpha \approx \frac{\pi}{2}$ wartość $C(\alpha)$ jest bardzo duża (zadanie jest źle uwarunkowane).\\
        Dla $\alpha \approx 0$ wartość $C(\alpha)$ jest bliska $1$ (zadanie jest dobrze uwarunkowane).
    \end{mdframed}

    \zad Zbadaj uwarunkowanie obliczania następujących funkcji:
    \begin{enumerate}[wide, labelindent=0pt]
        \item $f(x) = \cos(x)$,

              \begin{mdframed}
                  \[ C(x) = \biggl|\frac{x\cdot f'(x)}{f(x)}\biggr| = \biggl|\frac{-x\sin x}{\cos(x)}\biggr| = |x\tg x| \]
                  Dla $x \approx \frac{\pi}{2} + k\pi, k\in\Z$ wartość $C(x)$ jest bardzo duża (zadanie jest źle uwarunkowane).\\
                  Dla $x \approx k\pi, k\in\Z$ wartość $C(x)$ jest bliska $0$, błędy zostaną stłumione.
              \end{mdframed}

              \pagebreak
        \item $f(x) = \ln x$.

              \begin{mdframed}
                  \[ C(x) = \biggl|\frac{x\cdot f'(x)}{f(x)}\biggr| = \biggl|\frac{x\cdot\frac{1}{x}}{\ln x}\biggr| = \biggl|\frac{1}{\ln x}\biggr| \]
                  Dla $x \approx 1$ wartość $C(x)$ jest bardzo duża (zadanie jest źle uwarunkowane).\\
                  Dla $x \approx 0$ wartość $C(x)$ jest bliska $0$, błędy zostaną stłumione.\\
                  \[ \lim_{x\to\pm\infty} C(x) = 0\]
                  Dla $|x - 1| \gg 0$ zadanie jest dobrze uwarunkowane.
              \end{mdframed}
    \end{enumerate}

    \zad Zbadaj uwarunkowanie obliczania następujących funkcji:
    \begin{enumerate}[wide, labelindent=0pt]
        \item $f(x) = x^3+8$,

              \begin{mdframed}
                  \[ C(x) = \biggl|\frac{x\cdot f'(x)}{f(x)}\biggr| = \biggl|\frac{3x^3}{x^3+8}\biggr| \]
                  Dla $x \approx -2$ wartość $C(x)$ jest bardzo duża (zadanie jest źle uwarunkowane dla $x \approx -2$).\\
                  Dla $x \approx 0$ wartość $C(x)$ jest bliska $0$, błędy zostaną stłumione.\\
                  \[ \lim_{x\to\pm\infty} C(x) = 3\]
                  Dla $|x+2| \gg 0$ wartość $C(x)$ jest bliska $3$ (zadanie jest dobrze uwarunkowane).
              \end{mdframed}

        \item $f(x) = |x|$.

              \begin{mdframed}
                  Dla $x > 0$:
                  \[ C(x) = \biggl|\frac{x\cdot f'(x)}{f(x)}\biggr| = \biggl|\frac{x\cdot 1}{x}\biggr| = 1 \]
                  Obliczenie wartości funkcji $f$ jest dobrze uwarunkowane dla dowolnego $x > 0$.

                  \noindent Dla $x < 0$:
                  \[ C(x) = \biggl|\frac{x\cdot f'(x)}{f(x)}\biggr| = \biggl|\frac{x\cdot -1}{-x}\biggr| = 1 \]
                  Obliczenie wartości funkcji $f$ jest dobrze uwarunkowane dla dowolnego $x < 0$.

                  \noindent Dla $x = 0$ wartość funkcji jest po prostu równa $0$, \emph{obliczanie uwarunkowania w~punkcie (?) nie ma sensu}.\\

                  \mdfsubtitle{Ewentualnie z definicji:}
                  \[ \left| \frac{|x| - |x(1+\delta)|}{|x|}\right| = \left| \frac{\cancel{|x|} - \cancel{|x|}(1+\delta)}{\cancel{|x|}}\right| = \underbrace{1}_{C(X)}\cdot|\delta| \]
              \end{mdframed}
    \end{enumerate}
\end{lista}

\end{document}